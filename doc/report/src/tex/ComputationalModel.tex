\section{Computational model}

        Il modello migliore è stato ottenuto dopo una prima fase di \textit{data preparation}, ed una successiva fase di \textit{hyper-parameter tuning}.

        \subsection{Data preparation}
        
                \subsubsection{Data splitting}
                
                        Il \textit{dataset} è stato diviso in due sezioni differenti:
                        \begin{itemize}
                                \item \textit{training set}: contiene 80\% dell'intero dataset originale (6400 records)
                                \item \textit{test set}: contiene il restante 20\% del dataset originale (1600 records)
                        \end{itemize}
                
                \subsubsection{Bad values management}
                
                        Il \textit{dataset} originale (\textit{training\_set.csv}) contiene dati mancanti in corrispondenza di alcune delle \textit{features}, per questo motivo, i valori relativi a tali campi sono stati sostituiti con la mediana relativa alla colonna della \textit{feature} corrispondente all'interno del \textit{dataset}.
                        \bigbreak
                        
                        In seguito, sono stati analizzati gli \textit{outliers}. Un \textit{outlier} è un'osservazione che devia marcatamente da altre osservazioni nel campione di dati. L'identificazione di potenziali \textit{outliers} è importante dal momento che potrebbero indicare dati corrotti o mal codificati, anche se, in alcuni casi, potrebbero essere il risultato di variazioni casuali nel campione di dati.
                        
                        L'individuazione degli \textit{outliers} dipende dalla distribuzione dei dati sottostante. Come si evince dal pair-plot in figura \ref{fig:training_set_pairplot}, le \textit{features} del \textit{dataset} considerato, seguono approssimativamente una distribuzione \textit{Normale}, pertanto sono stati analizzati i risultati derivanti da tre metodi:
                        \begin{itemize}
                            \item \textit{z-score}
                            \item \textit{modified z-score}
                            \item \textit{inter-quartile range}
                        \end{itemize}
                    
                        Lo \textit{Z-score} di una osservazione è definito come:
                        \begin{displaymath}
                                z_i = \frac{x_i - \mu_x}{\sigma_x}
                        \end{displaymath}
                        essendo $\mu_x$ e $\sigma_x$, rispettivamente, la media e la deviazione standard del campione di dati. Sono considerati \textit{outliers} i campioni con il valore assoluto dello score maggiore di $3$.
                        \smallbreak
                        
                        Lo \textit{Z-score modificato} (\textit{Iglewicz and Hoaglin}) è definito come:
                        \begin{displaymath}
                                m_i = \frac{0.6745 \cdot (x_i - \tilde{x})}{MAD}
                        \end{displaymath}
                        essendo $MAD$ la median absolute deviation e $\tilde{x}$ la mediana. Sono considerati \textit{outliers} i campioni con il valore assoluto dello score maggiore di $3.5$.
                        \smallbreak
                        
                        L'\textit{inter-quartile range} è definito come:
                        \begin{displaymath}
                                IQR = Q_3 - Q_1
                        \end{displaymath}
                        essendo $Q_1$ e $Q_3$, rispettivamente, il primo ed il terzo quartile del campione di dati. Sono considerati \textit{outliers} i campioni con uno valore maggiore di $Q_3 + IQR \cdot 1.5$ e minore di $Q_1 - IQR \cdot 1.5$.
                        \bigbreak
                        
                        Il metodo \textit{modified z-score} è risultato il migliore per l'individuazione di \textit{outliers}. Infine, per non ridurre la numerosità del \textit{training set} si è scelto di sostituire (e non eliminare) gli \textit{outliers} con la mediana relativa ad una determinata \textit{feature}. Si è ritenuto opportuno utilizzare la mediana in quanto è una misura più robusta per la rappresentazione della dispersione di valori rispetto alla media, essendo meno soggetta alla presenza di \textit{outliers}.                        
                
                \subsubsection{Data normalization}
                
                La normalizzazione è il processo di ridimensionamento dei singoli campioni per avere una norma unitaria. Questo processo può essere utile se si prevede di utilizzare una forma quadratica per quantificare la somiglianza di qualsiasi coppia di campioni.
                
                La normalizzazione, con eventuale scaling dei valori nel range $[0, 1]$, è stata effettuata utilizzando i seguenti \textit{scalers}: \textit{MinMaxScaler, MaxAbsScaler, QuantileTransformer (uniform output), PowerTransformer (Yeo-Johnson e Box-Cox transforms)}.
                
        		
        		\begin{center}
        			La tabella \ref{tab:fscoreMLMinMax} mostra i risultati ottenuti con MinMaxScaler\\
                \begin{tabular}{+l^c^c}
                	\toprule\rowstyle{\bfseries}
                	Model & Parameters & F1-score \\
                	\toprule
                	
                	Multi-Layer Perceptron & \makecell[l]{activation: relu \\ hidden\_layer\_sizes: (100, 50) \\ learning\_rate: adaptive \\ learning\_rate\_init: 0.01 \\ solver: sgd} & \textbf{0.886} \\
                	\midrule
                	Support Vector Machine & \makecell[l]{C: 10 \\ decision\_function\_shape: ovo \\ gamma: 10 \\ kernel: rbf} & 0.794 \\
                	\midrule
                	Decision Tree & \makecell[l]{criterion: entropy \\ max\_depth: 90 \\ max\_features: None \\ min\_samples\_leaf: 1 \\ min\_samples\_split: 2 \\ splitter: best} & 0.607 \\
                	\midrule
                	Random Forest & \makecell[l]{criterion: entropy \\ max\_depth: 90 \\ max\_features: log2 \\ min\_samples\_leaf: 2 \\ min\_samples\_split: 2 \\ n\_estimators: 400} & 0.808 \\
                	\midrule
                	K-Nearest Neighbors & \makecell[l]{metric: minkowski \\ n\_neighbors: 3 \\ p: 4} & 0.768 \\
                	\midrule
                	Ada Boost &  \makecell[l]{class\_weight: balanced \\ max\_depth: 90 \\ max\_features: 3 \\ min\_samples\_leaf: 4 \\ n\_estimators: 400} & 0.823 \\
                	\midrule
                	Naive Bayes & \makecell[l]{priors: None \\ var\_smoothing: 10e-08} & 0.55 \\
                	\midrule
                	K-Means & \makecell[l]{max\_iter: 10000 \\ n\_clusters: 4 \\ random\_state: 43531} & 0.344 \\
                	
                	\toprule\rowstyle{\bfseries}
                \end{tabular}
                            \captionof{table}{\textit{F1-score} con sampler SMOTE e scaler MinMaxScaler}
            	\label{tab:fscoreMLMinMax}
		        \end{center}
	        
	        
	        
		        \begin{center}
		        	La tabella \ref{tab:f1scoreMLmaxAbs} mostra i risultati ottenuti con MaxAbsScaler\\
		        	\begin{tabular}{+l^c^c}
		        		\toprule\rowstyle{\bfseries}
		        		Model & Parameters & F1-score \\
		        		\toprule
		        		
		        		Multi-Layer Perceptron & \makecell[l]{activation: relu \\ hidden\_layer\_sizes: (150, 100) \\ learning\_rate: adaptive \\ learning\_rate\_init: 0.1 \\ solver: sgd} & \textbf{0.84} \\
		        		\midrule
		        		Support Vector Machine & \makecell[l]{C: 10 \\ decision\_function\_shape: ovo \\ gamma: 10 \\ kernel: rbf} & 0.692 \\
		        		\midrule
		        		Decision Tree & \makecell[l]{criterion: entropy \\ max\_depth: 90 \\ max\_features: None \\ min\_samples\_leaf: 1 \\ min\_samples\_split: 2} & 0.625 \\
		        		\midrule
		        		Random Forest & \makecell[l]{criterion: entropy \\ max\_depth: 80 \\ max\_features: sqrt \\ min\_samples\_leaf: 2 \\ min\_samples\_split: 2 \\ n\_estimators: 400} & 0.813 \\
		        		\midrule
		        		K-Nearest Neighbors & \makecell[l]{metric: minkowski \\ n\_neighbors: 3 \\ p: 3} & 0.76 \\
		        		\midrule
		        		Ada Boost &  \makecell[l]{class\_weight: balanced \\ max\_depth: 90 \\ max\_features: 3 \\ min\_samples\_leaf: 4 \\ n\_estimators: 300} & 0.815 \\
		        		\midrule
		        		Naive Bayes & \makecell[l]{priors: None \\ var\_smoothing: 0.01} & 0.549 \\
		        		\midrule
		        		
		        		\toprule\rowstyle{\bfseries}
		        	\end{tabular}
	        	\captionof{table}{\textit{F1-score} con sampler SMOTE e scaler MaxAbsScaler}
	        	\label{tab:f1scoreMLmaxAbs}
		        \end{center}
	        
	        
	        
		        \begin{center}
		        	La tabella \ref{tab:f1scoreMLQuantile} mostra i risultati ottenuti con QuantileTransformer\\
		        	\begin{tabular}{+l^c^c}
		        		\toprule\rowstyle{\bfseries}
		        		Model & Parameters & F1-score \\
		        		\toprule
		        		
		        		Multi-Layer Perceptron & \makecell[l]{activation: relu \\ hidden\_layer\_sizes: (150, 100) \\ learning\_rate: adaptive \\ learning\_rate\_init: 0.1 \\ solver: sgd} & \textbf{0.828} \\
		        		\midrule
		        		Support Vector Machine & \makecell[l]{C: 10 \\ decision\_function\_shape: ovo \\ gamma: 10 \\ kernel: rbf} & 0.743 \\
		        		\midrule
		        		Decision Tree & \makecell[l]{criterion: entropy \\ max\_depth: 90 \\ max\_features: None \\ min\_samples\_leaf: 5 \\ min\_samples\_split: 2} & 0.616 \\
		        		\midrule
		        		Random Forest & \makecell[l]{criterion: entropy \\ max\_depth: 80 \\ max\_features: log2 \\ min\_samples\_leaf: 2 \\ min\_samples\_split: 2 \\ n\_estimators: 500} & 0.813 \\
		        		\midrule
		        		K-Nearest Neighbors & \makecell[l]{metric: minkowski \\ n\_neighbors: 11 \\ p: 3} & 0.809 \\
		        		\midrule
		        		Ada Boost &  \makecell[l]{class\_weight: balanced \\ max\_depth: 90 \\ max\_features: 3 \\ min\_samples\_leaf: 4 \\ n\_estimators: 200} & 0.818 \\
		        		\midrule
		        		Naive Bayes & \makecell[l]{priors: None \\ var\_smoothing: 0.01} & 0.492 \\
		        		\midrule
		        		
		        		\toprule\rowstyle{\bfseries}
		        	\end{tabular}
	        		        	\captionof{table}{\textit{F1-score} con sampler SMOTE e scaler QuantileTransformer}
	        	\label{tab:f1scoreMLQuantile}
		        \end{center}
	        
	        
	       
				 \begin{center}
					La tabella \ref{tab:f1scoreMLPowerTransformerYJ} mostra i risultati ottenuti con PowerTransformer (Yeo-Johnson)\\
					\begin{tabular}{+l^c^c}
						\toprule\rowstyle{\bfseries}
						Model & Parameters & F1-score \\
						\toprule
						
						Multi-Layer Perceptron & \makecell[l]{activation: relu \\ hidden\_layer\_sizes: (150, 100) \\ learning\_rate: adaptive \\ learning\_rate\_init: 0.1 \\ solver: sgd} & \textbf{0.821} \\
						\midrule
						Support Vector Machine & \makecell[l]{C: 50 \\ decision\_function\_shape: ovo \\ gamma: 10 \\ kernel: rbf} & 0.78 \\
						\midrule
						Decision Tree & \makecell[l]{criterion: entropy \\ max\_depth: 80 \\ max\_features: None \\ min\_samples\_leaf: 2 \\ min\_samples\_split: 2} & 0.616 \\
						\midrule
						Random Forest & \makecell[l]{criterion: entropy \\ max\_depth: 90 \\ max\_features: sqrt \\ min\_samples\_leaf: 2 \\ min\_samples\_split: 2 \\ n\_estimators: 200} & 0.813 \\
						\midrule
						K-Nearest Neighbors & \makecell[l]{metric: minkowski \\ n\_neighbors: 3 \\ p: 3} & 0.767 \\
						\midrule
						Ada Boost &  \makecell[l]{class\_weight: balanced \\ max\_depth: 90 \\ max\_features: 3 \\ min\_samples\_leaf: 4 \\ n\_estimators: 200} & 0.829 \\
						\midrule
						Naive Bayes & \makecell[l]{priors: None \\ var\_smoothing: 1e-08} & 0.553 \\
						\midrule
						
						\toprule\rowstyle{\bfseries}
					\end{tabular}
       		        	\captionof{table}{\textit{F1-score} con sampler SMOTE e scaler PowerTransformer (Yeo-Johnson)}
						\label{tab:f1scoreMLPowerTransformerYJ}
				\end{center}
        
        
        
        				 \begin{center}
			        	La tabella \ref{tab:f1scoreMLPowerTransformerBC} mostra i risultati ottenuti con PowerTransformer (Box-Box)\\
			        	\begin{tabular}{+l^c^c}
			        		\toprule\rowstyle{\bfseries}
			        		Model & Parameters & F1-score \\
			        		\toprule
			        		
			        		Multi-Layer Perceptron & \makecell[l]{activation: relu \\ hidden\_layer\_sizes: (150, 100) \\ learning\_rate: adaptive \\ learning\_rate\_init: 0.1 \\ solver: sgd} & \textbf{0.882} \\
			        		\midrule
			        		Support Vector Machine & \makecell[l]{C: 50 \\ decision\_function\_shape: ovo \\ gamma: 10 \\ kernel: rbf} & 0.78 \\
			        		\midrule
			        		Decision Tree & \makecell[l]{criterion: entropy \\ max\_depth: 80 \\ max\_features: None \\ min\_samples\_leaf: 2 \\ min\_samples\_split: 2} & 0.622 \\
			        		\midrule
			        		Random Forest & \makecell[l]{criterion: entropy \\ max\_depth: 80 \\ max\_features: log2 \\ min\_samples\_leaf: 2 \\ min\_samples\_split: 2 \\ n\_estimators: 400} & 0.812 \\
			        		\midrule
			        		K-Nearest Neighbors & \makecell[l]{metric: minkowski \\ n\_neighbors: 3 \\ p: 3} & 0.767 \\
			        		\midrule
			        		Ada Boost &  \makecell[l]{class\_weight: balanced \\ max\_depth: 90 \\ max\_features: 3 \\ min\_samples\_leaf: 4 \\ n\_estimators: 300} & 0.824 \\
			        		\midrule
			        		Naive Bayes & \makecell[l]{priors: None \\ var\_smoothing: 1e-08} & 0.553 \\
			        		\midrule
			        		
			        		\toprule\rowstyle{\bfseries}
			        	\end{tabular}
			        	\captionof{table}{\textit{F1-score} con sampler SMOTE e scaler PowerTransformer (Box-Cox)}
			        	\label{tab:f1scoreMLPowerTransformerBC}
			        \end{center}
        
                
                \subsubsection{Feature selection}
                
                        In questa fase sono state selezionate le \textit{features} più significative al fine di ridurre i costi di training e migliorare la capacità di generalizzazione del classificatore.
                        Per "filtrare" le features viene assegnato un punteggio ad ogni feature utilizzando una funzione infine vengono rimosse tutte tranne le k feature con il punteggio maggiore.
                   
                        
                        \begin{figure}[!h]
                            \centering
                            \includegraphics[width=175mm]{training_set}
                            \caption{Pair plot representing \textit{features}.}
                            \label{fig:training_set_pairplot}
                        \end{figure}
                        \clearpage
                
                \subsubsection{Data sampling} 
                
                        Il \textit{dataset} considerato presenta elementi così distribuiti:
                        \begin{itemize}
                                \item \textit{Class 1}: $33.67 \, \%$
                                \item \textit{Class 2}: $15.99 \, \%$
                                \item \textit{Class 3}: $20.66 \, \%$
                                \item \textit{Class 4}: $29.68 \, \%$
                        \end{itemize}
                        
                        \`E stato quindi effettuato balancing delle classi per evitare che il modello finale pecchi nel riconoscimento di classi meno presenti nel \textit{training set}.
                        
                        La tecnica di balancing utilizzata è lo \textit{SMOTE}.
                        
                        \textbf{TODO}
                

        \subsection{Hyper-parameters tuning}
        
                I modelli utilizzati per la classificazione sono:
                \begin{itemize}
                        \item \textit{Multi-Layer Perceptron}
                        \item \textit{Support Vector Machine}
                        \item \textit{Decition Tree}
                        \item \textit{Random Forest}
                        \item \textit{K-Nearest Neighbors}
                        \item \textit{Stochastic Gradient Descent}
                        \item \textit{Ada Boost}
                        \item \textit{Naive Bayes}
                        \item \textit{K-Means}
                \end{itemize}
            
                In questa fase sono stati cercati gli \textit{iper-parametri} migliori per i vari classificatori tra quelli elencati nella seguente tabella.
                
                \begin{center}
                	\begin{tabular}{+l^c^c}
                		\toprule\rowstyle{\bfseries}
                		Model & Parameters \\
                		\toprule
                		
                		Multi-Layer Perceptron & \makecell[l]{activation: [tanh, relu] \\ hidden\_layer\_sizes: [(150, 100), (120, 60), (60, 30), (75), (45)] \\ learning\_rate: [constant, adaptive] \\ learning\_rate\_init: [1e-1, 1e-2, 1e-3, 1e-4] \\ solver: [sgd, adam]}  \\
                		\midrule
						 Support Vector Machine & \makecell[l]{kernel: linear \\ C: [0.1, 1, 10] \\ decision\_function\_shape: [ovo, ovr]\\ \midrule kernel: rbf \\ C: [0.1, 1, 10] \\ decision\_function\_shape: [ovo, ovr] \\ gamma: [1e-4, 1e-3, 1e-2, 1e-1, 1e+1, 1e+2, 1e+3, 1e+4] \\
						 \midrule kernel: poly \\ C: [0.1, 1, 10] \\ degree: [2, 3, 4]\\ decision\_function\_shape: [ovo, ovr] \\ gamma: scale} \\
						 \midrule
						 Decision Tree & \makecell[l]{criterion: [entropy, gini] \\ max\_depth: [80, 90] \\ max\_features: [log2, sqrt, None] \\ min\_samples\_leaf: [2, 5, 10] \\ splitter: [best, random]} \\
						 \midrule
						 Random Forest & \makecell[l]{criterion: [entropy, gini] \\ max\_depth: [80, 90] \\ max\_features: [log2, sqrt, None] \\ min\_samples\_leaf: [2, 5, 10] \\ n\_estimators: [100, 200, 300, 400, 500]} \\
						 \midrule
						 K-Nearest Neighbors & \makecell[l]{metric: [minkowski, euclidean, chebyshev] \\ n\_neighbors: [3, 5, 7, 11] \\ p: [3, 4, 5]} \\
						 \midrule
						 Ada Boost &  \makecell[l]{criterion: [entropy, gini] \\ class\_weight: balanced \\ max\_depth: 90 \\ max\_features: 3 \\ min\_samples\_leaf: 4 \\ n\_estimators: [100, 200, 300]\\ splitter: [best, random]} \\
						 \midrule
						 Naive Bayes & \makecell[l]{priors: [None, [0.25, 0.25, 0.25, 0.25]] \\ var\_smoothing: [10e-9, 10e-6, 10e-3, 10e-1]} \\
						 \midrule
						 K-Means & \makecell[l]{max\_iter: 10000 \\ n\_clusters: 4 }\\
 
                		\toprule\rowstyle{\bfseries}
                	\end{tabular}
                	\captionof{table}{Lista degli iperparametri testati per ogni modello di \textit{machine learning} testati}
                	\label{tab:f1scoreML}
                \end{center}
                

        \subsection{Evaluation}
        
                La metrica utilizzata al fine della valutazione dei classificatori è la \textit{f1-score} (media armonica), definita come segue:
                \begin{displaymath}
                F1 = \frac{2 \cdot precision \cdot recall}{precision + recall}
                \end{displaymath}
                con,
                \begin{displaymath}
                precision = \frac{TP}{TP + FP} \qquad recall = \frac{TP}{TP + FN}
                \end{displaymath}
                essendo $TN$ il numero di veri negativi, $FP$ il numero di falsi positivi, $FN$, falsi negativi e $TP$ veri positivi.