\section{Computational model}

        Il modello migliore è stato ottenuto dopo una prima fase di \textit{data preparation}, ed una successiva fase di \textit{hyper-parameter tuning}.

        \subsection{Data preparation}
        
                \subsubsection{Data splitting}
                
                        Il \textit{dataset} è stato diviso in due sezioni differenti:
                        \begin{itemize}
                                \item \textit{training set}: contiene 80\% dell'intero dataset originale (6400 records)
                                \item \textit{test set}: contiene il restante 20\% del dataset originale (1600 records)
                        \end{itemize}
                
                \subsubsection{Bad values management}
                
                        Il \textit{dataset} originale contiene dati mancanti in corrispondenza di alcune delle \textit{features}, per questo motivo, i valori relativi a tali campi sono stati sostituiti con il valore medio della \textit{feature} all'interno del \textit{dataset}.
                        \bigbreak
                        
                        Per la gestione degli \textit{outlier} sono stati analizzati i risultati derivanti da tre metodi:
                        \begin{itemize}
                            \item \textit{z-score}
                            \item \textit{modified z-score}
                            \item \textit{inter-quantile range}
                        \end{itemize}
                        
                        \textbf{TODO - descrivere in formule gli approcci dei 3 metodi}
                        
                        Per non ridurre la numerosità del \textit{training set} si è scelto di sostituire gli \textit{outliers} con la mediana relativa ad una determinata \textit{feature}. Si è ritenuto opportuno utilizzare la mediana in quanto è una misura più robusta per la rappresentazione della dispersione di valori rispetto alla media essendo meno soggetta alla presenza di \textit{outliers}.                        
                
                \subsubsection{Data normalization}
                
                \subsubsection{Feature selection}
                
                        In questa fase sono state selezionate le \textit{features} più significative al fine di ridurre i costi di training e migliorare la capacità di generalizzazione del classificatore.
                        
                        \begin{figure}[!h]
                            \centering
                            \includegraphics[width=175mm]{training_set}
                            \caption{Pair plot representing \textit{features}.}
                            \label{fig:training_set_pairplot}
                        \end{figure}
                        \clearpage
                
                \subsubsection{Data sampling}  
                
                

        \subsection{Hyper-parameters tuning}
        
                In questa fase sono stati cercati gli \textit{iper-parametri} per i vari classificatori.