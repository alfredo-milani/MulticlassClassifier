\section{Conclusions}

        Dai risultati ottenuti si evince che il miglior modello per la classificazione multi-classe relativamente al \textit{dataset} fornito, è il \textbf{\textit{Multi-Layer Perceptron}} con parametri [activation='relu', hidden\_layer\_sizes=(150, 100), learning\_rate='adaptive', learning\_rate\_init=0.01, max\_iter=10000, solver='sgd')], utilizzando la mediana di ogni \textit{feature} nella gestione di valori mancanti, \textit{modified z-score} per l'individuazione di \textit{outliers}, scaler di tipo \textit{BEST\_SCALER}, \textit{SelectKBest} per \textit{feature selection}, \textit{SMOTE} sampling.
        \bigbreak
        
        Una possibile miglioria da applicare al modello potrebbe quella di aumentare il numero di percettroni in ogni livello, ma questo comporterebbe un significativo aumento della complessità computazionale portando conseguentemente ad un notevole aumento del tempo di esecuzione nel training del modello.